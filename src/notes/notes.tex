\documentclass{tufte-handout}
\usepackage{amsmath}

% Set up the images/graphics package
\usepackage{graphicx}
\setkeys{Gin}{width=\linewidth,totalheight=\textheight,keepaspectratio}
\graphicspath{{figures/}}

\title{Notes about the NGCC-PCC-DAC model}

\author{dthierry}

% Make prettier tables.
\usepackage{booktabs}

% The units package provides nice, non-stacked fractions and better spacing
% for units.
\usepackage{units}

% The fancyvrb package lets us customize the formatting of verbatim
% environments.  We use a slightly smaller font.
\usepackage{fancyvrb}
\fvset{fontsize=\normalsize}

% Small sections of multiple columns
\usepackage{multicol}
\usepackage{amssymb}



%%% Custom Commands
%---------------------------------------------------------------------------
% Concise referencing

\begin{document}

\maketitle

\begin{abstract}
\noindent abstract
\end{abstract}



\section{Introduction}
%---------------------------------------------------------------------------

Little notes.

\section{Section 1: Disjunctive formulation}
%---------------------------------------------------------------------------

\begin{equation}
	\bigvee_{m \in M}
	\begin{pmatrix} 
    Y_m \\
		\sum_{i \in I_m} \lambda_{m,i} \left(k \right) x_{m, i} = \text{Pr} \left(k\right)
		\\ 
		\sum_{i \in I_m} \lambda_{m,i} \left(k \right) = 1 \\
    0 \leq \lambda_{m,i} \left(k \right) \leq 1 \quad \forall i \\
    A_m w_{m}\left(k\right) = b_m
	\end{pmatrix}, \; k \in {0,...,T},
\end{equation}

Variables $\text{Pr}\left(k \right)$, $\lambda_{m,i} \left(k \right)$

Suppose the vector of variables that are directly dependent on the load factor (e.g., fuel) can be calculated with a linear equation in the form $w \left( k \right)= A \text{Load}\left(k \right) + b$. Then, if it is possible to calculate the coefficients of $A$ and $b$ for each operation mode $m \in M$, the disjunctive constraint for the operating modes can be written as follows, 
\begin{equation}
	\bigvee_{m \in M}
	\begin{bmatrix} 
    Y_m \left(k\right)\\
		\text{Load} \left(k\right) = \sum_{i \in I_m} \lambda_{m,i} \left(k \right) \nu_{m, i}
		\\ 
		\sum_{i \in I_m} \lambda_{m,i} \left(k \right) = 1 \\
    0 \leq \lambda_{m,i} \left(k \right) \leq 1 \quad \forall i \\
    w_m = A_m \text{Load} + b_m
	\end{bmatrix}, \; k \in {0,...,T},
\end{equation}
where $Y_m \in \left\{\text{True, False} \right\}$ is a boolean variable and it is assumed that $\veebar_{m \in M} Y_m \left(k\right)$

\subsection{Subsection 1: MILP form}
\begin{equation}
\begin{split}
\sum_{I_m} \lambda_{m, i} \left(k\right) x_{m, i} &= \overline{\text{Pr}}_m \left(k \right) \\
\sum_{I_m} \lambda_{m, i} \left(k\right) & = y_m \left(k\right) \\
\sum_{m \in M} y_m \left(k\right) & = 1
\end{split}
\end{equation}



\begin{equation}
\begin{split}
	A_m \nu_m \left(k\right) &= b_m \\
	\nu_m \left(k\right) & = y_{m} \left(k\right)
\end{split}
\end{equation}

\subsection{Subsection 2: 2 $\times$ 2 $\times$ 1 of the GT-HRSG-ST train}
Since we have two Gas Turbines (GT), 2 HRSG and 1 Steam Turbine (ST), in principle, it is possible to shutdown a GT-HRSG
to reach half-ish the load. The only benefit of this is that we do can still use the same parameters. 
Even though this is a counter-intuitive way of modelling, we will offset the whole situation by having a single GT-HRSG active at all times.

New equation:
\begin{gather*}
\bigvee_{m \in M}
  \begin{bmatrix} 
    Y_m \left(k\right)\\
    \text{Load} \left(k\right) = \sum_{i \in I_m} \lambda_{m,i} \left(k \right) \nu_{m, i}
    \\ 
    \sum_{i \in I_m} \lambda_{m,i} \left(k \right) = 1 \\
    0 \leq \lambda_{m,i} \left(k \right) \leq 1 \quad i\in I_m \\
    w \left(k \right) = a_m \; \text{Load}\left(k \right) + b_m
  \end{bmatrix}, \; k \in {0,...,T}, \\
\Omega \left(Y_m \left( k\right) \right) = \text{True} \\
\veebar_{m \in M} Y_m \left(k\right) \\
Y_m \left(k \right) \in \left\{\text{True, False} \right\}, \\
0 \leq \text{Load}\left(k \right) \leq 100, \; w\left(k\right) \in \mathbb{R}^{n},\; k \in {0,...,T}
\end{gather*}


\begin{gather*}
\bigvee_{m \in M}
  \begin{bmatrix} 
    Y_{u,m} \left(k\right)\\
    \text{Load}_u \left(k\right) = \sum_{i \in I_m} \lambda_{u,m,i} \left(k \right) \nu_{u, m, i}
    \\ 
    \sum_{i \in I_m} \lambda_{u, m,i} \left(k \right) = 1 \\
    0 \leq \lambda_{u, m,i} \left(k \right) \leq 1 \quad i\in I_m \\
    w_u \left(k \right) = a_m \; \text{Load}_u \left(k \right) + b_m
  \end{bmatrix}, \\
 \qquad u \in \mathcal{U}, \; k \in {0,...,T}, \\
\Omega \left(Y_{u,m} \left( k\right) \right) = \text{True} \\
\veebar_{m \in M} Y_{u,m} \left(k\right) \\
Y_{u, m} \left(k \right) \in \left\{\text{True, False} \right\}, \\
0 \leq \text{Load}_u \left(k \right) \leq 100, \; w_u \left(k\right) \in \mathbb{R}^{n},\\
u \in \mathcal{U}, \; k \in {0,...,T}
\end{gather*}



\begin{gather*}
\text{Load}\left(k\right) = \sum_{u \in \mathcal{U}} \text{Load}_u\left(k\right) \\
\tilde{w} \left(k \right) = a \; \text{Load} \left(k \right) + b
\end{gather*}

\subsection{State-dependent transitions}

We encounter the following situation, if transition from mode $m$ to mode $m'$ occurs at time $k$, then transition from mode
$m''$ to $m$ must occur within the time window $\left[T^{L}_{m, m', m''}, T^{U}_{m, m', m''} \right]$. In other words,
\begin{equation}
    Z_{u,m,m'} \left(k\right) \Rightarrow \bigvee_{\theta = T^{L}_{m, m', m''}-1, \dots, T^{U}_{m, m', m''}} Z_{u, m'', m}\left(k - \theta\right).
\end{equation}
Which can be reformulated as follows:
\[
    \neg Z_{u, m, m'} \left(k \right)\vee \bigvee_{\theta = T^{L}_{m, m', m''}-1, \dots, T^{U}_{m, m', m''}} Z_{u, m'', m}\left(k - \theta\right).
\]

\[
    1 - z_{u, m, m'}\left(k\right) + \sum_{\theta = T^{L}_{m, m', m''}-1}^{T^{U}_{m, m', m''}} z_{u, m'', m}\left(k - \theta\right) \geq 1
\]
\[
    z_{u, m, m'}\left(k\right) \leq \sum_{\theta = T^{L}_{m, m', m''}-1}^{T^{U}_{m, m', m''}} z_{u, m'', m}\left(k - \theta\right)
\]
I want to have the opposite situation, i.e., 
\begin{equation}
    \neg z_{u,m,m'} \left(k\right) \Leftarrow \bigvee_{\theta = T^{L}_{m, m', m''}-1, \dots, T^{U}_{m, m', m''}} z_{u, m'', m}\left(k-\theta \right).
\end{equation}
This constraint can be reformulated as follows:
\[
    \neg \left[\bigvee_{\theta = T^{L}_{m, m', m''}-1, \dots, T^{U}_{m, m', m''}} z_{u, m'', m}\left(k-\theta\right) \right] \vee \neg z_{u, m, m'}\left(k\right)
\]
\[
    \bigwedge_{\theta = T^{L}_{m, m', m''}-1, \dots, T^{U}_{m, m', m''}} \left[ \neg z_{u, m'', m}\left(k- \theta \right) \vee \neg z_{u, m, m'}\left(k\right)\right]
\]
This last expression is equivalent to the following inequalities:
\[
    1-z_{u, m'', m} \left( k - \theta \right) + 1-z_{u, m, m'} \left( k \right) \geq 1, \quad \theta \in \{ T^{L}_{m, m', m''}-1, \dots, T^{U}_{m, m', m''} \}
\]    
\[
    z_{u, m'', m} \left( k - \theta \right) + z_{u, m, m'} \left( k \right)  \leq 1, \quad \theta \in \{ T^{L}_{m, m', m''}-1, \dots, T^{U}_{m, m', m''} \}
\]    
\subsection{Minimum stay constraint}
\begin{equation}
    \bigvee_{\theta = 0, \dots, K_{u,m,m'}^{\text{min}} -1} Z_{u, m, m'}\left(k-\theta \right) \Rightarrow Y_{u, m'} \left( k \right)
\end{equation}
\[
    \neg \left[\bigvee_{\theta = 0, \dots, K_{u,m,m'}^{\text{min}} -1} Z_{u, m, m'}\left(k-\theta \right) \right] \vee Y_{u, m'} \left(k \right)
\]
\[
    \bigwedge_{\theta = 0, \dots, K_{u,m,m'}^{\text{min}} -1} \left( \neg Z_{u, m, m'}\left(k-\theta \right) \vee Y_{u, m'} \left(k \right)
\right)
\]
\[
    1 - z_{u, m, m'}\left(k-\theta \right) + y_{u, m'} \left(k \right)\geq 1 \quad \theta \in \{ 0, \dots, K_{u,m,m'}^{\text{min}} -1\}  
\]
\[
    z_{u, m, m'}\left(k-\theta \right) \leq  y_{u, m'} \left(k \right) \quad \theta \in \{ 0, \dots, K_{u,m,m'}^{\text{min}} -1\}  
\]
\[
    y_{u, m'} \left(k \right) \geq z_{u, m, m'}\left(k-\theta \right) 
    \quad \theta \in \{ 0, \dots, K_{u,m,m'}^{\text{min}} -1\}  
\]
Which is not exactly what I had in mind but anyways. 


Here is the opossite situation:
\begin{equation}
    Y_{u, m'} \left( k \right) \Rightarrow \bigvee_{\theta = 0, \dots, K_{c,m,m'}^{\text{min}} -1} z_{u, m, m'}\left(k-\theta \right) 
\end{equation}

\[
    \neg 
    Y_{u, m'} \left( k \right) \vee  \bigvee_{\theta = 0, \dots, K_{c,m,m'}^{\text{min}} -1} z_{u, m, m'}\left(k-\theta \right) 
\]
\[
    1- y_{u, m'} \left( k \right) +  \sum_{\theta = 0}^{K_{c,m,m'}^{\text{min}} -1} z_{u, m, m'}\left(k-\theta \right) \geq 1
\]
\[
    y_{u, m'} \left( k \right) \leq  \sum_{\theta = 0}^{K_{c,m,m'}^{\text{min}} -1} z_{u, m, m'}\left(k-\theta \right)
\]

Finally, the two sided statement. 

\begin{equation}
    Y_{u, m} \left( k \right) \Leftrightarrow 
    \bigvee_{\theta=0, \dots, K^{\min}_{u, m, m'}-1} Z_{u, m, m'} \left(k - \theta \right)
\end{equation}

\subsection{Switch variables variables}
(Double Imp)
\[
Y_1 \Leftrightarrow Y_2
\]

\[
1-y_1 + y_2 \geq 1
\]
\[
y_1 \leq y_2
\]

\[
y_1 + 1 - y_2 \geq 1
\]

\[
y_1 \geq y_2
\]

(Actual constraint)
\[Y_{u, m} \left( k \right) = \left\{ \text{True, False} \right\}\]
\noindent $Y_{u, m} \left( k \right)=$True if mode $m$ is active at time $k$ for unit $u$  

\[
    Z_{u, m, m'}\left(k \right) = \left\{\text{True, False}\right\}
\]

\noindent $Z_{u, m, m'}\left(k \right)=$True if transition from $m$ to $m'$ occurs from time $k-1$ to $k$

\[
    Y_{u, m} \left( k \right) = \left\{Z_{u, m', m}\left(k\right),\dots  \right\}
\]
\[
    Y_{u, m} \left( k \right) \Leftrightarrow \bigoplus_{m' \in M} Z_{u, m', m}\left(k\right)
\]

\[
    \neg Y_{u, m} \left( k \right) \vee \bigoplus_{m' \in M} Z_{u, m', m} \left(k \right)
\]

\[
    Y_{1}\left(k\right) \Leftrightarrow   Z_{0, 1} \left( k \right) \oplus Z_{2, 1} \left( k \right)
\]

\[
    Y_{0}\left(k-1\right) \Leftrightarrow   Z_{0, 1} \left( k \right) \oplus Z_{0, 2} \left( k \right)
\]

\[
    \begin{split}
        Y_{0}\left(k\right) &\Leftrightarrow   Z_{1, 0} \left( k \right) \oplus Z_{2, 0} \left( k \right) \\
        Y_{1}\left(k\right) &\Leftrightarrow   Z_{0, 1} \left( k \right) \oplus Z_{2, 1} \left( k \right) \\
        Y_{2}\left(k\right) &\Leftrightarrow   Z_{0, 2} \left( k \right) \oplus Z_{1, 2} \left( k \right) \\
    \end{split}
\]

\[
    \begin{split}
        Y_{0}\left(k-1\right) &\Leftrightarrow   Z_{0, 1} \left( k \right) \oplus Z_{0, 2} \left( k \right) \\
        Y_{1}\left(k-1\right) &\Leftrightarrow   Z_{1, 0} \left( k \right) \oplus Z_{1, 2} \left( k \right) \\ 
        Y_{2}\left(k-1\right) &\Leftrightarrow   Z_{2, 0} \left( k \right) \oplus Z_{2, 1} \left( k \right) \\ 
    \end{split}
\]

\[
    y_{u,m}\left(k\right) = \sum_{m' \in M} z_{u, m', m} \left( k \right)
\]


\[
    Y_{u, m} \left( k-1 \right) \Leftrightarrow \bigoplus_{m' \in M} Z_{u, m, m'} \left(k\right)
\]

\[
    Y_{u, m} \left( k-1 \right) = \left\{Z_{u, m, m'}\left(k\right), \dots \right\}
\]

\[
    y_{u,m}\left(k-1 \right) = \sum_{m' \in M} z_{u, m, m'} \left( k \right)
\]

\sectin{Sync times}
I think there must be a separate set of constraints for this.

\begin{equation}
    \begin{pmatrix}
        W_{\text{cold}} \\
        
    \end{pmatrix}
    \vee
    \begin{pmatrix}
        W_{\text{warm} \\
    \end{pmatrix}
\end{equation}

%---------------------------------------------------------------------------
\end{document}
